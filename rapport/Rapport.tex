\documentclass[11pt,twoside,a4paper]{report}
\usepackage{a4}
\usepackage[ansinew]{inputenc}
\usepackage[danish]{babel}
\usepackage{textcomp}
\usepackage{paralist}		% for: \begin{compactitem}
\usepackage{subfigure}
\usepackage{paralist}		% for: \begin{compactitem}

\usepackage{graphicx}
\usepackage{fancyvrb} %centreret verbatim

\usepackage{listings, color}

\definecolor{forestgreen}{RGB}{34,139,34}
\definecolor{orangered}{RGB}{239,134,64}
\definecolor{darkblue}{rgb}{0.0,0.0,0.6}
\definecolor{gray}{rgb}{0.4,0.4,0.4}

\lstdefinestyle{XML} {
    language=XML,
    extendedchars=true, 
    breaklines=true,
    breakatwhitespace=true,
    emph={},
    emphstyle=\color{red},
    basicstyle=\ttfamily\small,
    columns=fullflexible,
    commentstyle=\color{gray}\upshape,
    morestring=[b]",
    morecomment=[s]{<?}{?>},
    morecomment=[s][\color{forestgreen}]{<!--}{-->},
    keywordstyle=\color{orangered},
    stringstyle=\ttfamily\color{black}, %\normalfont,
    tagstyle=\color{darkblue}\bf\normalfont,
    morekeywords={attribute,xmlns,version,type,release,name,xmlns:xsi,xmlns:xsd,minOccurs,maxOccurs,nillable,ref},
}


\lstdefinestyle{FAKE_XQUERY} {
    language=XML,
    alsolanguage=XSLT,
    extendedchars=true, 
    breaklines=true,
    breakatwhitespace=true,
    emph={},
    emphstyle=\color{red},
    basicstyle=\ttfamily\small,
    columns=fullflexible,
    commentstyle=\color{gray}\upshape,
    morestring=[b]",
    morecomment=[s]{<?}{?>},
    morecomment=[s][\color{forestgreen}]{<!--}{-->},
    keywordstyle=\color{orangered},
    stringstyle=\ttfamily\color{black}, %\normalfont,
    tagstyle=\color{darkblue}\bf\normalfont,
}


\lstdefinelanguage{XQUERY}
{
morekeywords={declare, function, string, as, dateTime, element, for, in, fn, return},
sensitive=false,
morestring=[b]",
}



\usepackage{fancyhdr}
\pagestyle{fancyplain} %til at lave overskrift p� hver side
%\setlength{\parindent}{0pt} %indrykning ved ny sektion
\fancyhf{} % delete current header and footer
\fancyhead[OL]{\leftmark}
\fancyhead[OR]{\thepage}
\fancyhead[EL]{\thepage}
\fancyhead[ER]{\leftmark} %giver overskrift \rightmark giver subsection overskrift
\fancypagestyle{plain}{%
\fancyhead{} % get rid of headers on plain pages
\renewcommand{\headrulewidth}{0pt} % and the line
}\pagestyle{fancy}
%ops�tning af TOC
\usepackage{hyperref} %make linkable
\hypersetup{
    colorlinks,
    citecolor=black,
    filecolor=black,
    linkcolor=black,
    urlcolor=black
}
\usepackage[toc,page]{appendix}

\setcounter{tocdepth}{2} %TOC only covers down to subsections
\setcounter{secnumdepth}{2} %section er det nederste niveau der tildeles nummer

\newcommand{\fct}[1]{\textit{#1}} %formatering for funktioner
\newcommand{\tbl}[1]{\textit{#1}} %formatering for tabeller
\newcommand{\flt}[1]{\emph{#1}} %formatering for felter
%\newcommand{\cmd}[1]{\textit{#1}} %formatering for entiteter
%\newcommand{\primkey}[1]{\underline{#1}} %formatering for primary key
%\newcommand{\schema}[1]{\emph{#1}} %formatering for skema

\title{Kompleks data og logik i databasen\\Brug af XML dokumenter til en videoapplikation}
\author{Stefan Jaensch, J�rgen Bo Arp Ladekj�r}
\date{Januar 2014 - Marts 2014}
\begin{document}
\maketitle
\newpage
\tableofcontents
\newpage

\chapter{Introduktion}
Videoapplikationer som Netflix, Youbio og Viaplay er blevet meget popul�re i de seneste par �r. Dette projekt handler om at skabe nogle de s�gefunktioner, som der m� v�re brug for i s�danne videoapplikationer, hvor et stort antal videoer er tilg�ngelige for brugeren. Disse s�gefunktioner skal g�re det muligt for brugeren at navigere rundt i de mange videoer og finde netop det indhold eller den specifikke video, som m�tte have interesse for brugeren.  

Som datakilde til dette projekt er der valgt et �bent API (Application Programming Interface) fra Danmarks Radio, som stiller stort set alt deres egenproducerede indhold tilg�ngeligt for brugeren/programm�ren via adressen http://www.dr.dk/nu/api . Dette API giver fx mulighed for at hente en total oversigt over alle tilg�ngelige videoer, i et enkelt XML dokument, samt hente detaljer omkring specifikke videoer.

I dette projekt hentes disse XML dokumenter ud via en standard webbrowser og gemmes lokalt, hvorefter de indl�ses i en BaseX XML database, hvorfra der vil blive udarbejdet forskellige s�gefunktioner til foresp�rgsler i XML dokumenterne.

I �jeblikket findes der desv�rre ingen tilg�ngelig beskrivelse af struktur eller elementer i de tilg�ngelige XML dokumenter, s� derfor starter projektet med en analyse af netop strukturen eller elementerne i de XML dokumenter som er udvalgt til dette projekt.


\chapter{Unders�gelse af XML dokumenterne}\label{chapter:study-xml-documents}
I det �bne API fra Danmarks Radio er der mange forskellige XML dokumenter til r�delighed. Nogle af XML dokumenter er forholdsvis sm� og indeholder kun en sti til en grafik, som fx kan anvendes i en grafisk brugerflade. Til dette projekt er der udvalgt 2 store XML dokumenter, som indeholder beskrivelser af de enkelte videoer og den programserie som de evt. er en del af.

XML dokumenterne er hentet fra nedenst�ende adresser og deres struktur og indhold beskrives i underafsnittene her under.
\begin{itemize}
\item \textbf{http://www.dr.dk/nu/api/videos/all.xml}
\item \textbf{http://www.dr.dk/nu/api/programseries.xml}
\end{itemize}

\section{Beskrivelse af dokumentet all.xml}
Dette XML dokument indeholder information om alle videoer som er tilg�ngelige via API�et. Strukturen af dette dokument kan betragtes som v�rende relativ flad, da dybde i strukturen er relativ lille. Strukturen best�r udelukkende best�r af rodelementet ArrayOfProgramSerieVideo, som indeholder elementer af ProgramSerieVideo for hver tilg�ngelig video der findes. Selve elementet ProgramSerieVideo indeholder en r�kke elementer direkte under sig, som beskriver detaljerne omkring videoen. Disse detaljer er sidste niveau i strukturen, hvilket g�r at dybde af strukturen kan betragtes som v�rende flad. Skulle dybde �ges vil det fx v�re muligt ved at samle elementer som BitrateKbps, Height og Width under et nyt element, fx med navnet TechDetails. 

Omkring dokumentets indhold ses det at der er angivet en prolog, som fort�ller at dokumentet er XML version 1 og er skrevet med tegns�t UTF-8. Elementerne under elementet ProgramSerieVideo er ikke beskrevet yderligere i API�et fra Danmarks Radio, s� de er i projektet fortolket p� f�lgende m�de:
\begin{itemize}
\item \textbf{Id:} Et unikt id for hver video der findes tilg�ngelig.
\item \textbf{Description:} En beskrivende tekst af hver video, som typisk anvendes i en tv-program-guide.
\item \textbf{ProgramSerieSlug:} Et navn som er tildelt videoer som er en del af en serie af programmer. Dette element kan betragtes som en slags fremmen�gle til elementet slug i XML dokumentet programseries.
\item \textbf{Title:} En beskrivende overskrift til videoen.
\item \textbf{Duration:} Videoen l�ngde (ikke set anvendt i endnu)
\item \textbf{BroadcastTime:} Tidspunkt for f�rst gang videoen blev vist i tv.
\item \textbf{ExpireTime:} Tidspunkt hvorefter videoen ikke vil v�re tilg�ngelig mere.
\item \textbf{PublishTime:} Tidspunkt hvor videoen blev gjort tilg�ngelig.
\item \textbf{Expired:} Er videoen ikke til tilg�ngelig mere?
\item \textbf{BroadcastChannel:} Tv-kanal som videoen blev sendt p�.
\item \textbf{VideoManifestUrl:} Link til streamning af videoen.
\item \textbf{VideoResourceUrl:} Link til streamning af videoen.
\item \textbf{Premiere:} Er denne video en premierevideo?
\item \textbf{BitrateKbps:} Datahastighed ved streamning af videoen. (Ikke altid sat)
\item \textbf{Height:} Videoens h�jdeopl�sning i pixels. (Ikke altid sat)
\item \textbf{Width:} Videoens bredeopl�sning i pixels. (Ikke altid sat)
\end{itemize}

\begin{figure}[ht]
%\centering
\begin{lstlisting}[style=XML]
<?xml version="1.0" encoding="utf-8"?>
<ArrayOfProgramSerieVideo 
 xmlns:xsi="http://www.w3.org/2001/XMLSchema-instance" 
 xmlns:xsd="http://www.w3.org/2001/XMLSchema">
  <ProgramSerieVideo>
    <Id>5062</Id>
    <Description>Det allerf�rste So ein Ding program ser p� HP Touch Smart IQ 500. Men hvor h�j er denne computers IQ egentlig?</Description>
    <ProgramSerieSlug>so-ein-ding</ProgramSerieSlug>
    <Title>Touch Smart sk�rme � So ein Ding</Title>
    <Duration />
    <BroadcastTime>2009-02-03T20:30:00</BroadcastTime>
    <ExpireTime>3000-01-01T00:00:00</ExpireTime>
    <PublishTime>0001-01-01T00:00:00</PublishTime>
    <Expired>false</Expired>
    <BroadcastChannel>DR2</BroadcastChannel>
    <VideoManifestUrl>http://www.dr.dk/Forms/Published/PlaylistGen.aspx?qid=1946138&amp;OnlyWritePath=True</VideoManifestUrl>
    <VideoResourceUrl>http://www.dr.dk/handlers/GetResource.ashx?id=853642</VideoResourceUrl>
    <Premiere>false</Premiere>
    <BitrateKbps>0</BitrateKbps>
    <Height>0</Height>
    <Width>0</Width>
  </ProgramSerieVideo>
  <!-- Herefter kommer mange flere ProgramSerieVideo elementer -->
</ArrayOfProgramSerieVideo>
\end{lstlisting}
\caption{Eksempel p� dokumentet all.xml}
\label{data:all.xml}
\end{figure}

\section{Beskrivelse af dokumentet programseries.xml}

Dette XML dokument indeholder information om alle serier af programmer, som er tilg�ngelige via API�et. Ved serier forst�s programmer som er opdelt i mange afsnit. Strukturen af dette dokument er lidt dybere end dokumentet all.xml, men ellers er den overordnet struktur identisk. Rodelementet hedder nu ArrayOfProgramSerie og indeholder elementer af ProgramSerie, som beskriver detaljer omkring selve serien af det specifikke program. Der hvor dokumentet adskiller sig dybdem�ssigt i forhold til dokumentet all.xml er i elementet Labels som kan indeholde en til mange elementer af String, som er en kategorisering af seriens emne.

I dokumentets indhold ses en prolog, som er identisk for det forrige dokuments prolog. Igen er elementerne under elementet ProgramSerie ikke beskrevet yderligere i API�et fra Danmarks Radio, s� de er i projektet fortolket p� f�lgende m�de:

\begin{itemize}
\item \textbf{Slug:} Unik n�gle et den enkelte serie af programmer. Denne n�gle anvendes som en slags fremmen�gle i det forrige dokument (all.xml).
\item \textbf{Title:} En beskrivende overskrift til serien af programmet.
\item \textbf{Description:} En beskrivende tekst af hver serie, som typisk anvendes i en tv-program-guide.
\item \textbf{ShortName:} (Ikke set anvendt)
\item \textbf{NewestVideoId:} En slags fremmen�gle til id�et p� den nyeste video i dokumentet all.xml.
\item \textbf{NewestVideoPublishTime:} Tidspunkt for udgivelse af den nyeste video i serien af programmet.
\item \textbf{VideoCount:} Antal af videoer i denne serie af programmer.
\item \textbf{Labels:} Kategorisering af series indhold.
\item \textbf{String:} En kort beskrivende tekst til kategorisering under Labels.
\end{itemize}  

\begin{figure}[ht]
%\centering
\begin{lstlisting}[style=XML]
<?xml version="1.0" encoding="utf-8"?>
<ArrayOfProgramSerie 
 xmlns:xsi="http://www.w3.org/2001/XMLSchema-instance" 
 xmlns:xsd="http://www.w3.org/2001/XMLSchema">
  <ProgramSerie>
    <Slug>so-ein-ding</Slug>
    <Title>So ein Ding</Title>
    <Description>Det bliver ikke nemmere. Den allersidste Ding er ...</Description>
    <ShortName />
    <NewestVideoId>96149</NewestVideoId>
    <NewestVideoPublishTime>2014-01-09T11:39:28</NewestVideoPublishTime>
    <VideoCount>157</VideoCount>
    <Labels>
      <string>tech og viden</string>
    </Labels>
    <WebCmsImagePath />
  </ProgramSerie>
  <!-- Herefter kommer mange flere ProgramSerie elementer -->
</ArrayOfProgramSerie>
\end{lstlisting}
\caption{Eksempel p� dokumentet programseries.xml}
\label{data:programseries.xml}
\end{figure}


















TODO: Insert stuff about study xml documents

Description of the selected XML documents

What documents we will use for this project?

What is the structure of the XML document? 

What elements are there and what is described in the element?

Proposals for changes in the structure or elements?
(Is there anything that can be improved)
(Attributes might be used instead of some of the elements)

\chapter{Validering af XML dokumenterne}\label{chapter:xml-schemas}
Som beskrevet i det foreg�ende kapitel, s� er der ikke i API'et ikke udstillet noget validerings-skema eller nogen indbygget DTD (Document Type Definition) til de XML dokumenter der er til r�dighed. Den manglende mulighed for at validere XML dokumenterne betyder, at videoapplikation m� stole p� at XML dokumenterne altid overholder samme struktur. �ndres strukturen pludseligt vil det betyde at de s�gefunktioner, som udarbejdes i dette projekt, ikke vil kunne fungere l�ngere. Det er dermed sagt, at der kun gives garanti for at alle de funktioner, som udarbejdes i dette projekt kun kan anvendes s�fremt at de XML dokumenter, som hentes fra Danmarks Radio's API, kan valideres imod de XML skemaer, som beskrives i dette kapitel. 

Nogle af fordelene ved at anvende XML skema fremfor DTD er, at XML skema er meget mere kraftfuldt end DTD da XML skema eksempelvis underst�tter datatyper og desuden er selve XML skemaer selv beskrevet med XML Syntax og derfor er relative lette at l�se.

\section{Beskrivelse af XML skemaerne}
Ved unders�gelse af XML dokumenterne fremgik det at begge XML dokumenter indeholder en sekvens af enten videoer eller programserier. Derfor best�r f�rste del af begge de udarbejde XML skemaer af en sekvens af enten elementet ProgramSerieVideo eller elementet ProgramSerie, som kan forekomme uendeligt mange gange. B�de elementet ProgramSerieVideo og ProgramSerie er oprettet som et complexType element, der best�r af hver sine beskrivende elementer, som fx Slug, Title, Description osv. Disse elementer som beskriver videoen eller programserien er blevet sat til at v�re af en bestemt type. Valget af type til elementerne er fremkommet dels af elements navn og elements indhold, som har givet et hint om hvilke type der er tale om.

Specielt skal det bem�rkes at elementet �BroadcastTime� i �ProgramSerieVideo� er sat til at kunne forekommer i XML dokumentet med en null v�rdi (nillable). Det vil sige at en enkelt video kan have elementet � BroadcastTime � i sig, men uden nogen v�rdi. En anden speciel detalje er at elementet BroadcastChannel i �ProgramSerieVideo�, er sat til at kunne undlades helt i en video. Dette g�lder �vrigt ogs� ShortName, WebCmsImagePath i skemaet til ProgramSeries.

I skemaet til ProgramSeries er det ogs� v�rd at bem�rke element LabelsType, som er et complexType element, som indg�r i et andet complexType element, nemlig elementet ProgramSerieType.  

\begin{figure}[ht]
%\centering
\begin{lstlisting}[style=XML]
<?xml version="1.0"?>
<xs:schema xmlns:xs="http://www.w3.org/2001/XMLSchema">
	<xs:element name="ArrayOfProgramSerieVideo">
		<xs:complexType>
			<xs:sequence>
				<xs:element name="ProgramSerieVideo" 
				            maxOccurs="unbounded"
							type="ProgramSerieVideoType" />
			</xs:sequence>
		</xs:complexType>
	</xs:element>	
	<xs:complexType name="ProgramSerieVideoType">
		<xs:sequence>
			<xs:element name="Id" 					
				        type="xs:nonNegativeInteger" />
			<xs:element name="Description" 			
				        type="xs:string" />
			<xs:element name="ProgramSerieSlug" 	
				        type="xs:string" />
			<xs:element name="Title" 				
				        type="xs:string" />
			<xs:element name="Duration" 			
				        type="xs:string" />
			<xs:element name="BroadcastTime" 		
				        type="xs:dateTime" 
						nillable="true" />
			<xs:element name="ExpireTime" 			
				        type="xs:dateTime" />
			<xs:element name="PublishTime" 			
				        type="xs:dateTime" />
			<xs:element name="Expired" 				
				        type="xs:boolean" />
			<xs:element name="BroadcastChannel" 	
				        type="xs:string" 
						minOccurs="0" 
						maxOccurs="1" />
			<xs:element name="VideoManifestUrl" 	
				        type="xs:anyURI" />
			<xs:element name="VideoResourceUrl" 	
				        type="xs:anyURI" />
			<xs:element name="Premiere" 			
				        type="xs:boolean" />
			<xs:element name="BitrateKbps" 			
				        type="xs:nonNegativeInteger" />
			<xs:element name="Height" 				
				        type="xs:nonNegativeInteger" />
			<xs:element name="Width" 				
				        type="xs:nonNegativeInteger" />
		</xs:sequence>
	</xs:complexType>
</xs:schema>
\end{lstlisting}
\caption{XML skema til dokumentet all.xml}
\label{data:all_schema.xsd}
\end{figure}


\begin{figure}[ht]
%\centering
\begin{lstlisting}[style=XML]
<?xml version="1.0"?>
<xs:schema xmlns:xs="http://www.w3.org/2001/XMLSchema">
	<xs:element name="ArrayOfProgramSerie">
		<xs:complexType>
			<xs:sequence>
				<xs:element name="ProgramSerie" maxOccurs="unbounded" type="ProgramSerieType"/>
			</xs:sequence>
		</xs:complexType>
	</xs:element>
	<xs:complexType name="ProgramSerieType">
		<xs:sequence>
			<xs:element name="Slug" 			type="xs:string" />
			<xs:element name="Title" 			type="xs:string" />
			<xs:element name="Description" 		type="xs:string" />
			<xs:element name="ShortName" 		type="xs:string" minOccurs="0" maxOccurs="1" />
			<xs:element name="NewestVideoId"	type="xs:nonNegativeInteger" />
			<xs:element name="NewestVideoPublishTime" type="xs:dateTime" />
			<xs:element name="VideoCount"		type="xs:nonNegativeInteger" />
			<xs:element name="Labels"			type="LabelsType" />
			<xs:element name="WebCmsImagePath" type="xs:anyURI" minOccurs="0" maxOccurs="1" />
		</xs:sequence>
	</xs:complexType>
	<xs:complexType name="LabelsType">
		<xs:sequence>		
			<xs:element name="string"	type="xs:string" minOccurs="0" maxOccurs="unbounded" />
		</xs:sequence>	
	</xs:complexType>	
</xs:schema>
\end{lstlisting}
\caption{XML skema til dokumentet programseries.xml}
\label{data:programseries.xsd}
\end{figure}

\section{Validering med XML skema}

Valideringen af XML dokumenterne med XML skemaerne er test ved hj�lp af BaseX, hvor validate:xsd er blevet anvendt. For at test XML skemaerne holder over tid, s� er der igennem projektetsforl�b blevet hentet nye versioner af XML dokumenterne, s� ogs� har best�et valideringstesten med de udarbejde XML skemaer. 

\begin{figure}[ht]
%\centering
\begin{lstlisting}[style=FAKE_XQUERY, language=XQUERY]
validate:xsd('all.xml', 'all_schema.xsd')
\end{lstlisting}
\caption{Funktionskald til validering af all.xml med skemaet all\_schema.xsd}
\label{xschema: validering }
\end{figure}


\chapter{XQuery s�gning}\label{chapter:xquery-search}

\section{Find alle kategorierne til programserierne}
Brugeren af videoapplikationen skal have muligheden for at navigere imellem de forskellige programserieres kategorier, som der findes p� et givent tidspunkt. Funktionen i figur \ref{xquerySearch:getProgramSeriesCategory} bruger et XPath udryk til at finde alle kategori navnene i XML dokumentet og funktionen distinct-value() selekterer kun unik navne. En �for lykke� returnerer kategorinavne som et element, \ele{category}, og derudover returnerer funktionen getProgramSeriesCategory() et element, \ele{countCategory}, med antal af forskellige kategorier. 

\begin{figure}[ht]
%\centering
\begin{lstlisting}[style=FAKE_XQUERY, language=XQUERY]

declare function dr:getProgramSeriesCategory
  ( $programseriesFileName as xs:string) as element()*{
    
    let $progSerieLabels := distinct-values(doc($programseriesFileName)/ArrayOfProgramSerie/ProgramSerie/
		Labels/string/text())
    return (
      <countCategory>{count($progSerieLabels)}</countCategory>,
      (
      for $label in $progSerieLabels
        return <category>{$label}</category>
      )
    )
};

\end{lstlisting}
\caption{Funktion til at finde kategorinavne og samlet antal af kategorier.}
\label{xquerySearch:getProgramSeriesCategory}
\end{figure}


\section{Find antal programserier i en kategori}
N�r en bruger v�lger en kategori skal videoapplikationen vise antallet af programserier i kategorien og forskellige informationer om disse. I funktionen \ref{xquerySearch:programSeriesOfCategory} finder XPath udtrykket alle \ele{ProgramSerie} elementer, hvor elementet \ele{string} i elementet \ele{Labels} har den sammen v�rdi som den valgte kategori. Variablen \$progSeries indeholder en sekvens med alle fundende \ele{ProgramSerie} elementer og ikke kun elementet \ele{Labels} fordi /.. ved slutning af sekvensen v�lger for�ldre elementet, som er \ele{ProgramSerie}.  XQuery funktionen count() t�ller antallet af fundende \ele{ProgramSerie} elementer. Denne information og alle de fundende \ele{ProgramSerie} elementer bliver returneret i elementet \ele{programSeriesCategory}.

\begin{figure}[ht]
%\centering
\begin{lstlisting}[style=FAKE_XQUERY, language=XQUERY]

declare function dr:programSeriesOfCategory
  ( $programseriesFileName as xs:string, $category as xs:string ) as element()*{
    
    let $progSeries := doc($programseriesFileName)/ArrayOfProgramSerie/
    ProgramSerie/Labels[string=$category]/..
    return (
      <programSeriesCategory>
        <nameCategory>{$category}</nameCategory>
        <countProgSeries>{count($progSeries)}</countProgSeries>
        {
         for $progSerie in $progSeries
           return $progSerie
         }
      </programSeriesCategory>
    )
};

\end{lstlisting}
\caption{Funktion til at finde antallet af programserier i en bestemt kategori.}
\label{xquerySearch:programSeriesOfCategory}
\end{figure}

\section{Find nyeste videoer inden for en specifik kategori}

For at videoapplikationen kan pr�sentere detaljeret videoinformationer og de nyeste videoer, n�r en bruger har valgt en kategori, er det n�dvendigt at hente information fra XML dokumentet all.xml. Informationerne om de enkelte videoer ligger alts� ikke i det sammen XML dokument, hvor informationen om programserierne og kategorier er gemt. I dette tilf�lde findes kategori informationer i dokumentet programseries.xml og videoinformationen i dokumentet all.xml. Et unikt video- ID som hedder \ele{NewestVideoId} i dokumentet programseries.xml og \ele{Id} i dokumentet all.xml referere fra et specifikt \ele{programserie} element til et specifikt \ele{ProgramSerieVideo} element. Her vises sig en svaghed ved XML som ikke har nogle begr�nsninger p� dataindholdet (fx unik) og fremmen�gle referencer mellem elementerne. Faktisk er der ingen information i de to XML dokumenter eller deres XML skemaer, som dokumenterer denne reference imellem elementerne i de to dokumenter.
Til s�gning efter alle programserier i en specifik kategori eksisterer der allerede en funktion som hedder programSeriesOfCategory() som er vist i figur \ref{xquerySearch:programSeriesOfCategory}. Denne funktion blev kaldt i funktionen videosOfCategory() i figur \ref{xquerySearch:videosOfCategory}. Den resulterende sekvens indeholder alle fundende \ele{programserie} elementer, som igen har det s�gte tag \ele{NewestVideoId}. En �for lykke� behandler elementerne \ele{programserie} og et XPath udtryk selekterer kun det unik video ID, som blev gemt i variabelen \$videoId. I den samme �for lykke� bliver \$videoId brugt som et argument i en anden XPath s�gning, men denne gang i videodokumentet  all.xml. Resultatet af funktionens s�gningen er elemeter af \ele{programSerieVideo} med tilh�rende detaljeret informationer.


\begin{figure}[ht]
%\centering
\begin{lstlisting}[style=FAKE_XQUERY, language=XQUERY]

declare function dr:videosOfCategory
( $programseriesFileName as xs:string,  $programseriesVideoFileName as xs:string, $category as xs:string ) as element()*{

  let $programSeriesCategory := dr:programSeriesOfCategory($programseriesFileName, $category )
  return(
    for $videoId in $programSeriesCategory//ProgramSerie/NewestVideoId/text()
      return(
        let $programSerieVideo := doc($programseriesVideoFileName)//ProgramSerieVideo[Id = $videoId]
        return $programSerieVideo
      )
  )
};
\end{lstlisting}
\caption{Funktion til at finde nyeste videoer inden for en specifik kategori.}
\label{xquerySearch:videosOfCategory}
\end{figure}




\section{Find alle videoer inden for et specifikt tidsinterval}

Videoapplikationen skal have en s�gefunktion, hvor brugeren kan finde alle de programserievideoer, hvor udsendelsestidspunktet ligger inden for et specifikt tidsinterval. XQuery funktionen er vist i figur \ref{xquerySearch:programSeriesVideoBroadcastDateInterval} og tager imod navnet af XML videofilen, start og stop tidspunktet, som argumenter til funktionen. En �for lykke� l�ser tidsinformation i tagget \ele{BroadcastTime} af hvert element \ele{ProgramSerieVideo} og returnerer kun de \ele{ProgramSerieVideo} elementer, hvor udsendelsestidspunktet ligger inden for tidsintervallet. XQuery har to forskellige sammenligningsoperatorer, v�rdier sammenligningsoperatorer fx (le, ge) som er beregnet til at sammenligne enkelte v�rdier og generelle sammenligningsoperatorer fx (\textless=, =\textgreater) for at sammenligne sekvenser med flere v�rdier. I denne funktion blev der brugt den f�rste variant fordi det kun er enkelte v�rdier som skal sammenlignes. Det er n�dvendig at tage h�jde for at tagget \ele{BroadcastTime} kan v�re tomt. I denne situation vil cast fra null til xs:dateTime() giver en dynamisk fejl. Til at afhj�lpe dette problem bruges der en hj�lpefunktion, figur \ref{xquerySearch:getBroadCastTime} som kontrollerer om tagget er tomt og dermed returnere en dummy dato ("1900-01-01T00:00:00") ellers den fundende dato, til at opdatere den interne variabel \$broadCastTime.

\begin{figure}[ht]
%\centering
\begin{lstlisting}[style=FAKE_XQUERY, language=XQUERY]

declare function dr:programSeriesVideoBroadcastDateInterval
( $programseriesVideoFileName as xs:string, $dateStart as xs:dateTime, $dateStop as xs:dateTime) as element()*{
  for $progSerVideoElement in doc($programseriesVideoFileName)/ArrayOfProgramSerieVideo/ProgramSerieVideo
  let $broadCastTime := dr:getBroadCastTime($progSerVideoElement/BroadcastTime)
  where $broadCastTime ge $dateStart and $broadCastTime le $dateStop
  return $progSerVideoElement
}; 

\end{lstlisting}
\caption{Funktion til at finde videoer for et specifikt tidsinterval}
\label{xquerySearch:programSeriesVideoBroadcastDateInterval}
\end{figure}



\begin{figure}[ht]
%\centering
\begin{lstlisting}[style=FAKE_XQUERY, language=XQUERY]

declare function dr:getBroadCastTime
($broadCastTimeElement as element() ) as xs:dateTime{
  if( not($broadCastTimeElement/text()) ) then (
       (xs:dateTime("1900-01-01T00:00:00"))
    ) else (
      (xs:dateTime($broadCastTimeElement/text()))
    ) 
};

\end{lstlisting}
\caption{Funktion til at l�se BroadcastTime}
\label{xquerySearch:getBroadCastTime}
\end{figure}





%TODO: Insert stuff about xquery search

% Description of each search and how it is solved

% Done! Find programs between specific date intervals.
% Done! Find all kinds of labels for prorgams.
% Done! Find the number of programs for each label.

% Identisk! Find all kinds of broadcasting channels.
% Find the number of programs for each channel.

% Find all videos from a particular series of programs sorted by date.





\chapter{Fuld tekst s�gning}\label{chapter:full-text-search}
TODO: Insert stuff about full-text search

Using Full-text search in BaseX

Find relevant videos based on full text search.

Sorting results by relevance (score).

Creating and using stop-word list

Creating af word cloud
(image before stop-word list and after stop-word list)

\chapter{Join imellem XML dokumenter}\label{chapter:joining-data}
I dette kapitel beskrives forskellige funktioner, som alle har det tilf�ldes at de alle arbejder p� 2 forskellige XML dokumenter samtidigt eller 2 versioner af samme XML dokumenter, men af forskellige dato. 

\section{Antal tilg�ngelige afsnit i en programserie}
Det er ikke altid at alle afsnit af et specifikt program er tilg�ngeligt i en programserie. Eksempelvis vil Danmarks Radio kun have et begr�nset antal af afsnit, liggende tilg�ngeligt for brugerne, n�r det drejer sig om relativt kostbarer produktioner som dramaserier, der efterf�lgende skal kunne s�lges p� eksempelvis DVD. I disse tilf�lde vil eksempelvis kun de 2 sidste afsnit v�re tilg�ngelige i en kort periode.

Det er derfor interessant at f� udarbejdet en funktion, som kan returnere det totale antal af afsnit samt antallet af afsnit som er tilg�ngelige netop nu.

Til dette form�l skal elementet med navnet VideoCount, fra dokumentet programseries.xml, anvendes som det antal af afsnit, der total findes i en specifik programserie. Til at finde antallet af afsnit, som faktisk er tilg�ngelige netop nu, bruges aggregeringsfunktionen Count p� elementerne med navnet ProgramSerieVideo i dokumentet all.xml, hvor det angives som betingelse, at der kun skal medtage de elementer, som indeholder samme v�rdi i elementet ProgramSerieSlug som i elementet Slug fra dokumentet programseries.xml.  Desuden �nskes kun de videoer, som har elementet Expired sat til falsk, da det kunne t�nkes at en video ikke er blevet fjernet fra dokumentet all.xml, men at selve linket til videoen ikke er gyldig l�ngere. 



TODO: Insert stuff about joining data

Which XML documents to be joined and why?

Compare the same XML document different dates.
( Is something added,removed,updated ? list results)

\chapter{XML i PostgreSQL}\label{chapter:xml-in-postgresql}
I dette kapitel undes�ges hvordan funktioner til videoapplikationen ville se ud hvis de skulle hente deres data direkte fra tabeller og ikke fra en XML dokument.

Funktionernes restultet vil dog stadig v�re i form af et resultats�t beskrevet med XML.

\section{Design af tabeller}
I figur \ref{postgresql:ER-diagram} ses ER-diagrammet. 

\begin{figure}[ht]
  \centering
   \includegraphics[width=1.1\textwidth]{pic/ER_programserie.png}
   \caption{ER-diagram over tabel-design.}
   \label{postgresql:ER-diagram}
\end{figure}


\chapter{Konklusion}{chapter:conclusion}
I dette projekt kan det konkluderes at XML er et godt sprog til dataudveksling, hvor enkelte objekter, fx i form af detaljer omkring en video eller en programserie skal udveksles. Is�r ses XML sproget som v�rende godt til online verdenen, hvor der ligefrem l�gger op til at outputtet data direkte som HTML. 

XML kommer derimod i problemer n�r der er tale om at holde p� mange informationer med tilh�rende relationer. Her t�nkes fx p� problemstillingen omkring alle programserier i et dokument, og alle informationer omkring de tilh�rende videoer i et andet dokument. Dokumenterne kan uden problemer, indeholde alt informationen, men der findes igen metode ti l at sikre at relationerne imellem de to dokumenter rent faktisk er eksisterende. En video i XML dokumentet med detaljer om videoerne kan godt referere til en programserie i XML dokumentet omkring alle programserierne, men der er ikke nogen sikkerhed for programserien rent faktisk findes i dette dokument. Her savnes princippet med fremmen�gler fra den relationelle database. Til gengl�de er XML et meget l�sebart sprog for det menneskelige �je, hvor der kan hurtigt skabes et overblik over dataenes hierarkiske struktur, hvilket kan v�re meget vanskeligt n�r der ses p� tabeller i en relationel database.

Omkring XQuery, s� konkluders det at syntaksen er noget speciel og uvant i forhold til andre sprog. Specielt blev det bem�rket, at det ikke kan lade sig g�re at skrive en IF-s�tning uden, at der skal med tages en ELSE, da denne er p�kr�vet ogs� selvom den ikke skal have noget indhold.    

XML skema.( skal altid ske ved indl�sning) Dermed undg�es fejl (som dynamisk fejl) n�r data type �ndres.

Funktioner til oversigt og med specfik detajler til valg af video.



\newpage
\clearpage \thispagestyle{empty}
\begin{appendix}
\chapter{Installations guide}
\subsection{Fil beskrivelse}


\subsubsection{XML data dokuments}

\begin{itemize}
\item \textbf{all.xml} Video (ældre version) Indholder detaljeret information om alle videoer.
\item \textbf{all\_new.xml} Video (nyere version)
\item \textbf{programseries.xml} Programserie (ældre version) med informationer om alle tilgængelige serier.
\item \textbf{programseries\_new.xml} Programserie (nyere version)
\end{itemize}



\subsubsection{XML Skemaer}

\begin{itemize}
\item \textbf{all\_schema.xsd} Skema til all\_x.xml filer
\item \textbf{programseries\_schema.xsd} Skema til programseries\_x.xml filer
\end{itemize}


\subsubsection{XQuery og XPath forespørgsler}

\begin{itemize}
\item \textbf{dr\_api\_module.xqm} XQuery modul med de meste af forespørgsler
\item \textbf{dr\_api\_module\_usage.xq} Eksempel som viser hvordan modulen dr\_api\_module.xqm bruges.
\item \textbf{word\_cloud\_search.xq} Forespørgsel til ord sky funktionen.
\item \textbf{text\_search-Restaurant-laks.xq} Full text søgning in programserie.
\item \textbf{ful-text-search-SCORE-Restaurant-laks.xq} Full text søgning in programserie med score information.

\end{itemize}


\subsubsection{Supplerende filer}

\begin{itemize}
\item{stop-words.txt} List med brugte stop ord til full text søgning.
\end{itemize}



\subsection{BaseX konfiguration}
I projektet blev brugt BaseX 7.8.1.

Opret en ny database i BaseX med \textit{Database - New} kommando for hver fil, programseries.xml, programseries\_new.xml, all.xml og all\_new.xml. Det er vigtigt at i den \textit{Create Database} menu under \textit{Indexes} den Full-Text Index blev aktiveret, sprog dansk og stop-word.txt filen bliver valgt som vist på figur \ref{inst:BaseXIndex}.

\begin{figure}[h!]
  \centering
   \includegraphics[width=0.8\textwidth]{pic/BaseX-Index.png}
   \caption{BaseX - Full-Text index indstilling }
	\label{inst:BaseXIndex}
\end{figure}



\end{appendix}

\end{document} 