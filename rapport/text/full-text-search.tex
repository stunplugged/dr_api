I videoapplikation vil der v�re behov for at kunne s�ge efter videoer eller programserier ud fra specifikke ord, som kan v�re en del af titlen eller fx beskrivelsen, som er den del der indeholder med tekst.  

\section{Oprettelse af full text index}

I BaseX databasen er det muligt at oprette et full text index til sin XML database. Index�et vil bliver oprettet p� indhold af alle elementer i XML doukmentet og ikke kun p� et specifikt element i dokumentet. Sammenlignes dette med en relationel database vil det svare til at index�et ikke kun oprettes p� en specefik kolonne i en specifik tabel, men alle kolonner i alle tabeller i databasen. 

\subsection{Indstillinger til full text index}

Ved oprettelsen af index�et skal sproget angives, ellers s�ttes det default til engelsk. I denne videoapplikation vil sproget v�re dansk og det f� stor betydning n�r det teksterne i databasen uds�ttes for �Stemming� . Stemming er en slags ordbog. som kan reducere ord ned til deres mindste form. Det vil fx sige at et ord som �Bilerne� vil blive til �bil� f�r det gemmes som en v�rdi i index�et.  

\section{Full text s�gning}

Det skal i videoapplikationen v�re muligt at s�ge efter specifikke ord inden for en specifik kategori af programserier. Det vil sige at full text s�gning skal kombineres med en XQuery, hvor kun elementer, som indg�r i den efterspurgte kategori af programserier med tages og kun returneres som resultat hvis de ogs� kan opfylde full text s�gningen.

\begin{figure}[ht]
%\centering
\begin{lstlisting}[style=FAKE_XQUERY, language=XQUERY]
//ArrayOfProgramSerie/ProgramSerie/Labels[string='livsstil']/../Description[ft:contains(text(), ("restaurant", "laks"), { "mode": "any" })]/..
\end{lstlisting}
\caption{Implementationen af s�gning efter specifik kategori af programserier med specifikke ord som indhold.}
\label{fulltextSearch:programserierLivsstilOrd}
\end{figure}

\section{En sky af ord}

Til videoapplikationen �nskes en flot baggrundsbillede, som best�r af en sky af de mest brugte ord i alle elementerne som beskriver programserierne. Til at skabe s�dan et billede skal full text index�et unders�ges. Indholdet skal tages ud i en r�kkef�lge hvor de mest brugte ord kommer f�rst.

\subsection{Tilf�jelse af stop words}

Da ikke alle ord er lige interesant at have med i index�et, s� er det muligt at oprette en stop words list, som indeholder alle de ord som skal undtages n�r full text index�et oprettes.
