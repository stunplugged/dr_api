I dette kapitel unders�ges hvordan funktioner til videoapplikationen ville se ud, hvis ikke skulle hente deres data fra et XML dokument men i stedet hente dataene direkte fra tabeller i en relationel database. Resultatet af et funktionskald skal dog stadig v�re i form af et resultats�t beskrevet med XML.

Databasen som vil blive brugt i dette projekt er en PostgreSQL database og der arbejdes udelukkende med data fra XML dokumentet programseries.xml.

\section{Design af tabeller}

F�rste opgave er at designe nogle tabeller, som kan underst�tte indholdet i XML dokumentet. Den f�rste tabel, der skal bruges er en tabel til at holde p� selve XML dokumentet i sin komplette tilstand.  Til det form�l er tabellen �downloadedXML� oprettet. Denne tabel indeholder to kolonner, en til navnet p� XML dokumentet (docmentTitle) og en til selve indholdet af dokumentet (documentContent) som er af typen XML.

Fra unders�gelsen af programseries.xml, tilbage i kapitel 2.2, er det allerede kendt at der findes en 0 til mange relation imellem en programserie og programseriens labels, alts� kategorier. I figur \ref{postgresql:ER-diagram} ses et ER-diagram, som viser denne relation.  Derfor er der oprettet en tabel udelukkende til labels og en tabel udelukkende til indholdet af en programserie. Til at beskrive relationen, at en programserie har 0 til mange label�s og at label�s har 0 til mange programserier, s� er der oprettet en tabel, som hedder slugLabel, der har en reference til label tabellen og en reference ti l programserie tabellen. Dermed en denne relation beskrevet i databasen og tabellerne er nu klar til at indl�st data i sig.

\begin{figure}[ht]
  \centering
   \includegraphics[width=1.1\textwidth]{pic/ER_programserie.png}
   \caption{ER-diagram over tabel-design.}
   \label{postgresql:ER-diagram}
\end{figure}

\section{Oprettelse af label�s}

N�r XML dokumentet er indl�st i tabellen downloadedXML, s� skal der som det f�rste oprettes labels til alle de kategorier, som findes i dokumentet programseries.xml. Til dette form�l er funktionen � getAndInsertAllLabels()�, som ses i figur \ref{ code: getAndInsertAllLabels} skabt. Denne funktion benytter et XPath-udtryk til at finde alle tekster til labels og for hver tekst den finder, s� unders�ges det om der allerede er oprettet en label med den tekst, hvis ikke s� inds�ttes denne nye label i tabellen. 

\begin{figure}[h]
\centering
\begin{BVerbatim}
  labels := (SELECT xpath('//ProgramSerie/Labels/string/text()', 
             downloadedXML.documentContent) FROM downloadedXML);
  FOR i IN array_lower(labels, 1) .. array_upper(labels, 1)
  LOOP
    SELECT labelNo INTO v_found FROM label WHERE label.labelTxt = labels[i];
    IF v_found IS NULL THEN
      INSERT INTO label(labelTxt) VALUES(labels[i]);
    END IF;
  END LOOP;
\end{BVerbatim}
\caption{Udsnit af funktionen getAndInsertAllLabels()}
\label{code: getAndInsertAllLabels }
\end{figure}


