
\section{Find alle kategorierne til programserierne}
Brugeren af videoapplikationen skal have muligheden for at navigere imellem de forskellige programserieres kategorier, som der findes p� et givent tidspunkt. Funktionen i figur \ref{xquerySearch:getProgramSeriesCategory} bruger et XPath udryk til at finde alle kategori navnene i XML dokumentet og funktionen distinct-value() selekterer kun unik navne. En �for lykke� returnerer kategorinavne som et element, \ele{category}, og derudover returnerer funktionen getProgramSeriesCategory() et element, \ele{countCategory}, med antal af forskellige kategorier. 

\begin{figure}[ht]
%\centering
\begin{lstlisting}[style=FAKE_XQUERY, language=XQUERY]

declare function dr:getProgramSeriesCategory
  ( $programseriesFileName as xs:string) as element()*{
    
    let $progSerieLabels := distinct-values(doc($programseriesFileName)/ArrayOfProgramSerie/ProgramSerie/
		Labels/string/text())
    return (
      <countCategory>{count($progSerieLabels)}</countCategory>,
      (
      for $label in $progSerieLabels
        return <category>{$label}</category>
      )
    )
};

\end{lstlisting}
\caption{Funktion til at finde kategorinavne og samlet antal af kategorier.}
\label{xquerySearch:getProgramSeriesCategory}
\end{figure}


\section{Find antal programserier i en kategori}
N�r en bruger v�lger en kategori skal videoapplikationen vise antallet af programserier i kategorien og forskellige informationer om disse. I funktionen \ref{xquerySearch:programSeriesOfCategory} finder XPath udtrykket alle \ele{ProgramSerie} elementer, hvor elementet \ele{string} i elementet \ele{Labels} har den sammen v�rdi som den valgte kategori. Variablen \$progSeries indeholder en sekvens med alle fundende \ele{ProgramSerie} elementer og ikke kun elementet \ele{Labels} fordi /.. ved slutning af sekvensen v�lger for�ldre elementet, som er \ele{ProgramSerie}.  XQuery funktionen count() t�ller antallet af fundende \ele{ProgramSerie} elementer. Denne information og alle de fundende \ele{ProgramSerie} elementer bliver returneret i elementet \ele{programSeriesCategory}.

\begin{figure}[ht]
%\centering
\begin{lstlisting}[style=FAKE_XQUERY, language=XQUERY]

declare function dr:programSeriesOfCategory
  ( $programseriesFileName as xs:string, $category as xs:string ) as element()*{
    
    let $progSeries := doc($programseriesFileName)/ArrayOfProgramSerie/
    ProgramSerie/Labels[string=$category]/..
    return (
      <programSeriesCategory>
        <nameCategory>{$category}</nameCategory>
        <countProgSeries>{count($progSeries)}</countProgSeries>
        {
         for $progSerie in $progSeries
           return $progSerie
         }
      </programSeriesCategory>
    )
};

\end{lstlisting}
\caption{Funktion til at finde antallet af programserier i en bestemt kategori.}
\label{xquerySearch:programSeriesOfCategory}
\end{figure}

\section{Find nyeste videoer inden for en specifik kategori}

For at videoapplikationen kan pr�sentere detaljeret videoinformationer og de nyeste videoer, n�r en bruger har valgt en kategori, er det n�dvendigt at hente information fra XML dokumentet all.xml. Informationerne om de enkelte videoer ligger alts� ikke i det sammen XML dokument, hvor informationen om programserierne og kategorier er gemt. I dette tilf�lde findes kategori informationer i dokumentet programseries.xml og videoinformationen i dokumentet all.xml. Et unikt video- ID som hedder \ele{NewestVideoId} i dokumentet programseries.xml og \ele{Id} i dokumentet all.xml referere fra et specifikt \ele{programserie} element til et specifikt \ele{ProgramSerieVideo} element. Her vises sig en svaghed ved XML som ikke har nogle begr�nsninger p� dataindholdet (fx unik) og fremmen�gle referencer mellem elementerne. Faktisk er der ingen information i de to XML dokumenter eller deres XML skemaer, som dokumenterer denne reference imellem elementerne i de to dokumenter.
Til s�gning efter alle programserier i en specifik kategori eksisterer der allerede en funktion som hedder programSeriesOfCategory() som er vist i figur \ref{xquerySearch:programSeriesOfCategory}. Denne funktion blev kaldt i funktionen videosOfCategory() i figur \ref{xquerySearch:videosOfCategory}. Den resulterende sekvens indeholder alle fundende \ele{programserie} elementer, som igen har det s�gte tag \ele{NewestVideoId}. En �for lykke� behandler elementerne \ele{programserie} og et XPath udtryk selekterer kun det unik video ID, som blev gemt i variabelen \$videoId. I den samme �for lykke� bliver \$videoId brugt som et argument i en anden XPath s�gning, men denne gang i videodokumentet  all.xml. Resultatet af funktionens s�gningen er elemeter af \ele{programSerieVideo} med tilh�rende detaljeret informationer.


\begin{figure}[ht]
%\centering
\begin{lstlisting}[style=FAKE_XQUERY, language=XQUERY]

declare function dr:videosOfCategory
( $programseriesFileName as xs:string,  $programseriesVideoFileName as xs:string, $category as xs:string ) as element()*{

  let $programSeriesCategory := dr:programSeriesOfCategory($programseriesFileName, $category )
  return(
    for $videoId in $programSeriesCategory//ProgramSerie/NewestVideoId/text()
      return(
        let $programSerieVideo := doc($programseriesVideoFileName)//ProgramSerieVideo[Id = $videoId]
        return $programSerieVideo
      )
  )
};
\end{lstlisting}
\caption{Funktion til at finde nyeste videoer inden for en specifik kategori.}
\label{xquerySearch:videosOfCategory}
\end{figure}




\section{Find alle videoer inden for et specifikt tidsinterval}

Videoapplikationen skal have en s�gefunktion, hvor brugeren kan finde alle de programserievideoer, hvor udsendelsestidspunktet ligger inden for et specifikt tidsinterval. XQuery funktionen er vist i figur \ref{xquerySearch:programSeriesVideoBroadcastDateInterval} og tager imod navnet af XML videofilen, start og stop tidspunktet, som argumenter til funktionen. En �for lykke� l�ser tidsinformation i tagget \ele{BroadcastTime} af hvert element \ele{ProgramSerieVideo} og returnerer kun de \ele{ProgramSerieVideo} elementer, hvor udsendelsestidspunktet ligger inden for tidsintervallet. XQuery har to forskellige sammenligningsoperatorer, v�rdier sammenligningsoperatorer fx (le, ge) som er beregnet til at sammenligne enkelte v�rdier og generelle sammenligningsoperatorer fx (\textless=, =\textgreater) for at sammenligne sekvenser med flere v�rdier. I denne funktion blev der brugt den f�rste variant fordi det kun er enkelte v�rdier som skal sammenlignes. Det er n�dvendig at tage h�jde for at tagget \ele{BroadcastTime} kan v�re tomt. I denne situation vil cast fra null til xs:dateTime() giver en dynamisk fejl. Til at afhj�lpe dette problem bruges der en hj�lpefunktion, figur \ref{xquerySearch:getBroadCastTime} som kontrollerer om tagget er tomt og dermed returnere en dummy dato ("1900-01-01T00:00:00") ellers den fundende dato, til at opdatere den interne variabel \$broadCastTime.

\begin{figure}[ht]
%\centering
\begin{lstlisting}[style=FAKE_XQUERY, language=XQUERY]

declare function dr:programSeriesVideoBroadcastDateInterval
( $programseriesVideoFileName as xs:string, $dateStart as xs:dateTime, $dateStop as xs:dateTime) as element()*{
  for $progSerVideoElement in doc($programseriesVideoFileName)/ArrayOfProgramSerieVideo/ProgramSerieVideo
  let $broadCastTime := dr:getBroadCastTime($progSerVideoElement/BroadcastTime)
  where $broadCastTime ge $dateStart and $broadCastTime le $dateStop
  return $progSerVideoElement
}; 

\end{lstlisting}
\caption{Funktion til at finde videoer for et specifikt tidsinterval}
\label{xquerySearch:programSeriesVideoBroadcastDateInterval}
\end{figure}



\begin{figure}[ht]
%\centering
\begin{lstlisting}[style=FAKE_XQUERY, language=XQUERY]

declare function dr:getBroadCastTime
($broadCastTimeElement as element() ) as xs:dateTime{
  if( not($broadCastTimeElement/text()) ) then (
       (xs:dateTime("1900-01-01T00:00:00"))
    ) else (
      (xs:dateTime($broadCastTimeElement/text()))
    ) 
};

\end{lstlisting}
\caption{Funktion til at l�se BroadcastTime}
\label{xquerySearch:getBroadCastTime}
\end{figure}





%TODO: Insert stuff about xquery search

% Description of each search and how it is solved

% Done! Find programs between specific date intervals.
% Done! Find all kinds of labels for prorgams.
% Done! Find the number of programs for each label.

% Identisk! Find all kinds of broadcasting channels.
% Find the number of programs for each channel.

% Find all videos from a particular series of programs sorted by date.



