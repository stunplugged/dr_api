
\section{Find alle kategorierne til programserierne}
Brugeren af videoapplikationen skal have muligheden for at navigere igennem de forskellige 
programseries kategorierne der findes p� tidspunktet. Funktionen i Figur \ref{xquerySearch:getProgramSeriesCategory} bruger en XPath udryk til at finde alle kategori navne i filen og funktionen distinct-value() selekterer kun unik navne. En for lykke returnerer kategorinavne som element med tag \textless category\textgreater og udover returnerer funktionen getProgramSeriesCategory() et element, \textless countCategory\textgreater, med antal af forskellige kategorier. 


\begin{figure}[ht]
%\centering
\begin{lstlisting}[style=FAKE_XQUERY, language=XQUERY]

declare function dr:getProgramSeriesCategory
  ( $programseriesFileName as xs:string) as element()*{
    
    let $progSerieLables := distinct-values(doc($programseriesFileName)/ArrayOfProgramSerie/ProgramSerie/Labels/
    	string/text())
    return (
      <countCategory>{count($progSerieLables)}</countCategory>,
      (
      for $lable in $progSerieLables
        return <category>{$lable}</category>
      )
    )
};

\end{lstlisting}
\caption{Funktion til at finde navn af programserierne og samlet antal.}
\label{xquerySearch:getProgramSeriesCategory}
\end{figure}



\section{Find antal programserier pr. kategori}
N�r en bruger v�gler et kategoriet skal videoapplikationen viser antallet af programserier i kategoriet og forskellige informationer om de. I funktionen \ref{xquerySearch:programSeriesOfCategory} finder den XPath udryk alle ProgramSerie elementer hvor element \textless string\textgreater i element \textless Labels\textgreater har den sammen v�rdi som det valgte kategori. Variablen \$progSeries holder en sekvens med alle fundet ProgramSerie elementer og ikke kun elementet \textless Labels\textgreater fordi /.. ved slutning af sekvensen v�lger for�ldre elementet som er ProgramSerie. Den XQuery funktion count() taler antal af fundet ProgramSerie elementer. Denne information og alle de fundet ProgramSerie elementer bliver returneret i elementet \textless programSeriesCategory>.

\begin{figure}[ht]
%\centering
\begin{lstlisting}[style=FAKE_XQUERY, language=XQUERY]

declare function dr:programSeriesOfCategory
  ( $programseriesFileName as xs:string, $category as xs:string ) as element()*{
    
    let $progSeries := doc($programseriesFileName)/ArrayOfProgramSerie/
    ProgramSerie/Labels[string=$category]/..
    return (
      <programSeriesCategory>
        <nameCategory>{$category}</nameCategory>
        <countProgSeries>{count($progSeries)}</countProgSeries>
        {
         for $progSerie in $progSeries
           return $progSerie
         }
      </programSeriesCategory>
    )
};

\end{lstlisting}
\caption{Funktion til at finde alle programserierne af kategoriet og samlet antal.}
\label{xquerySearch:programSeriesOfCategory}
\end{figure}



\section{Find alle videoer med i en specifik kategori}

For at pr�senterer detaljeret video informationer n�r en bruger har valgt et kategoriet er det n�dvendig at hende denne information fra xml dokumentet. Video informationer ligger ikke i det sammen xml dokument hvor informationen om programserie og kategorier er gemmt. I denne eksempel findes kategori informationer i filen programseries.xml og video informationen i filen all.xml. En unik video ID som hedder \textless NewestVideoId\textgreater i filen programseries.xml og \textless Id\textgreater i filen all.xml referencerer fra et programserie element til en ProgramSerieVideo element. Her vises sig en svaghed af XML som ikke har nogle constrains p� data indholdet (f.eks. uniknes) og referencering mellem elementerne. Faktisk er der ingen information i begge to XML dokumenterne ellers XML skemaer som dokumenterer denne reference mellem begge to tags.
Til s�gning after alle programserier i en specifik kategori eksisterer allerede en funktion og hedder programSeriesOfCategory() og er vist i Figur \ref{xquerySearch:programSeriesOfCategory}. Denne funktion blev kald i funktionen videosOfCategory() i Figur \ref(xquerySearch:videosOfCategory). Den resulterende sekvens indholder alle fundet programserie elementer som igen har den s�gte tag \textless NewestVideoId\textgreater. En for lykke behandler de programserie elementer og en XPath udryk selekterer kun den unik video ID som blev gemt i en variable \$videoId. I den samme for lykke bliver \$videoId brugt som en argument i en anden XPath s�gning, men denne gang i video XML filen. Resultatet af s�gning er programserie video informationer og bliver returneret af funktionen.


\begin{figure}[ht]
%\centering
\begin{lstlisting}[style=FAKE_XQUERY, language=XQUERY]

declare function dr:videosOfCategory
( $programseriesFileName as xs:string,  $programseriesVideoFileName as xs:string, $category as xs:string ) as element()*{

  let $programSeriesCategory := dr:programSeriesOfCategory($programseriesFileName, $category )
  return(
    for $videoId in $programSeriesCategory//ProgramSerie/NewestVideoId/data()
      return(
        let $programSerieVideo := doc($programseriesVideoFileName)//ProgramSerieVideo[Id = $videoId]
        return $programSerieVideo
      )
  )
};
\end{lstlisting}
\caption{Funktion til at finde alle videoer i en specifik kategoriet.}
\label{xquerySearch:videosOfCategory}
\end{figure}




\section{Find alle videoer inden for et specifikt tidsinterval}

Videoapplikationen skal have en s�gefunktion hvor brugeren kan finde programserievideoer hvor udsendelse tidspunkt ligger inden for et specifikt tidsinterval. XQuery funktionen er vist i Figur \ref{xquerySearch:programSeriesVideoBroadcastDateInterval} og tager navnet af XML video filen, start og stop tidspunktet som argument. En for lykke l�ser tidsinformation i tagget \textless BroadcastTime\textgreater af hvert element ProgramSerieVideo og returnerer kun de ProgramSerieVideo elementer hvor udsendelse tidspunktet ligger inden for tidsintervallet. XQuery har to forskellige sammenligningsoperatorer, v�rdier sammenligningsoperatorer (f.eks. le, ge) som er beregnet til at sammenligne enkelte v�rdi og generelte sammenligningsoperatorer (f.eks. \textless =, =\textgreater) for at sammenligne sekvenser med flere v�rdier. I denne funktion blev brugt de f�rste variant fordi der er kun enkelte v�rdier som skal sammenlignes. Det er n�dvendig at tage h�jde for at tagget \textless BroadcastTime\textgreater kan v�re tomt. I denne situation vil cast fra null til xs:dateTime() giver en dynamisk fejl. XQuery er et funktionelt programmeringssprog og der medf�lges at variabler er uforanderlig. Derfor bruges en hj�lpefunktion, Figur \ref{xquerySearch:getBroadCastTime} som kontrollerer n�r tagget er tomt og dermed returnere en dummy date ("1900-01-01T00:00:00") ellers den fundet date, til at opdatere den interne variable \$broadCastTime.


\begin{figure}[ht]
%\centering
\begin{lstlisting}[style=FAKE_XQUERY, language=XQUERY]

declare function dr:programSeriesVideoBroadcastDateInterval
( $programseriesVideoFileName as xs:string, $dateStart as xs:dateTime, $dateStop as xs:dateTime) as element()*{
  for $progSerVideoElement in doc($programseriesVideoFileName)/ArrayOfProgramSerieVideo/ProgramSerieVideo
  let $broadCastTime := dr:getBroadCastTime($progSerVideoElement/BroadcastTime)
  where $broadCastTime ge $dateStart and $broadCastTime le $dateStop
  return $progSerVideoElement
}; 

\end{lstlisting}
\caption{Funktion til at finde videoer for et specifikt tidsinterval}
\label{xquerySearch:programSeriesVideoBroadcastDateInterval}
\end{figure}



\begin{figure}[ht]
%\centering
\begin{lstlisting}[style=FAKE_XQUERY, language=XQUERY]

declare function dr:getBroadCastTime
($broadCastTimeElement as element() ) as xs:dateTime{
  if( not($broadCastTimeElement/data()) ) then (
       (xs:dateTime("1900-01-01T00:00:00"))
    ) else (
      (xs:dateTime($broadCastTimeElement/data()))
    ) 
};

\end{lstlisting}
\caption{Funktion til at l�se BroadcastTime}
\label{xquerySearch:getBroadCastTime}
\end{figure}





%TODO: Insert stuff about xquery search

% Description of each search and how it is solved

% Done! Find programs between specific date intervals.
% Done! Find all kinds of labels for prorgams.
% Done! Find the number of programs for each label.

% Identisk! Find all kinds of broadcasting channels.
% Find the number of programs for each channel.

% Find all videos from a particular series of programs sorted by date.



