Videoapplikationer som Netflix, Youbio og Viaplay er blevet meget popul�re i de seneste par �r. Dette projekt handler om at skabe nogle de s�gefunktioner, som der m� v�re brug for i s�danne videoapplikationer, hvor et stort antal videoer er tilg�ngelige for brugeren. Disse s�gefunktioner skal g�re det muligt for brugeren at navigere rundt i de mange videoer og finde netop det indhold eller den specifikke video, som m�tte have interesse for brugeren.  

Som datakilde til dette projekt er der valgt et �bent API (Application Programming Interface) fra Danmarks Radio, som stiller stort set alt deres egenproducerede indhold tilg�ngeligt for brugeren/programm�ren via adressen http://www.dr.dk/nu/api . Dette API giver fx mulighed for at hente en total oversigt over alle tilg�ngelige videoer, i et enkelt XML dokument, samt hente detaljer omkring specifikke videoer.

I dette projekt hentes disse XML dokumenter ud via en standard webbrowser og gemmes lokalt, hvorefter de indl�ses i en BaseX XML database, hvorfra der vil blive udarbejdet forskellige s�gefunktioner til foresp�rgsler i XML dokumenterne. Til sidst i projektet vil der ogs� blive set p� at udl�ses og gemme informationerne fra et XML dokument, i en relationel database, som i dette projekt vil v�re PostgreSQL. Der vil ligeledes blive implementeret et par funktioner til, at hente dataene ud igen, fra den relationelle database og aflevere dem tilbage i form af XML.

I �jeblikket findes der desv�rre ingen tilg�ngelig beskrivelse af struktur eller elementer i de tilg�ngelige XML dokumenter, s� derfor starter projektet med en analyse af netop strukturen eller elementerne i de XML dokumenter som er udvalgt til dette projekt.
