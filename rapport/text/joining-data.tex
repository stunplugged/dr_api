I dette kapitel beskrives forskellige funktioner, som alle har det tilf�ldes at de alle arbejder p� 2 forskellige XML dokumenter samtidigt eller 2 versioner af samme XML dokumenter, men af forskellige dato. 

\section{Antal tilg�ngelige afsnit i en programserie}
Det er ikke altid at alle afsnit af et specifikt program er tilg�ngeligt i en programserie. Eksempelvis vil Danmarks Radio kun have et begr�nset antal af afsnit, liggende tilg�ngeligt for brugerne, n�r det drejer sig om relativt kostbarer produktioner som dramaserier, der efterf�lgende skal kunne s�lges p� eksempelvis DVD. I disse tilf�lde vil eksempelvis kun de 2 sidste afsnit v�re tilg�ngelige i en kort periode.

Det er derfor interessant at f� udarbejdet en funktion, som kan returnere det totale antal af afsnit samt antallet af afsnit som er tilg�ngelige netop nu.

Til dette form�l skal elementet med navnet VideoCount, fra dokumentet programseries.xml, anvendes som det antal af afsnit, der total findes i en specifik programserie. Til at finde antallet af afsnit, som faktisk er tilg�ngelige netop nu, bruges aggregeringsfunktionen Count p� elementerne med navnet ProgramSerieVideo i dokumentet all.xml, hvor det angives som betingelse, at der kun skal medtage de elementer, som indeholder samme v�rdi i elementet ProgramSerieSlug som i elementet Slug fra dokumentet programseries.xml.  Desuden �nskes kun de videoer, som har elementet Expired sat til falsk, da det kunne t�nkes at en video ikke er blevet fjernet fra dokumentet all.xml, men at selve linket til videoen ikke er gyldig l�ngere. 

For at skabe et resultatset, for en helt specifik programserie (Slug), som kun indeholder antal videoer total og antal tilg�ngelige videoer, for en helt specifike programserie, s� er der implementeret 2 funktioner, som hedder henholdsvis checkVideoCount og countVideoSlugs.

Funktionen checkVideoCount tager imod 3 argumenter, som alle er af datatype string. SlugName er navnet p� selve programserien fx �so-ein-ding�, de 2 �vrige argumenter er navnene p� XML filerne, som henholdsvis indeholder listen af programserier (programseriesFileName) og liste over alle de enkelte videoer (videoFileName). Funktionen starter med at bruge et XPath-udtryk til at finde f�rste del af resultats�ttet, som er indholdet af elementet VideoCount, fra listen af programserier. Derefter kaldes en hj�lpefunktion, som hedder countVideoSlugs, hvor aggregeringsfunktionen count anvendes til samment�lle antallet af afsnit, som findes i XML dokumentet, der indeholder alle videoer. Betingelse for at blive talt med er, som tidligere n�vnt, at elementet Expired skal v�re sat til v�rdien falsk. Hj�lpefunktionen sender resultatet retur som en integer til funktionen checkVideoCount. Det samlede resultatset returens nu som en sekvens best�ende af to elementer: totalVideoCount og availableVideoCount. For at kunne returnere en sekvens af elementer, er funktionen checkVideoCount defineret til at returnere element()* som betyder at 0 til mange elementer kan forventes returneret fra funktionen. I dette tilf�lde vil antallet af elementer dog altid v�re 2, hvoraf det ene element (totalVideoCount) kan indeholde en tom v�rdi, hvis der foresp�rges p� en programserie (SlugName), som ikke findes i listen over programserier.

\begin{figure}[ht]
\begin{lstlisting}[style=FAKE_XQUERY, language=XQUERY]
declare function dr:checkVideoCount
  ($slugName as xs:string, $programseriesFileName as xs:string, $videoFileName as xs:string ) as element()*{
    let $totalCount := doc($programseriesFileName)/ArrayOfProgramSerie/ProgramSerie[Slug = $slugName]/VideoCount/text()
    return( 
      <totalVideoCount>{$totalCount}</totalVideoCount>,
      <availableVideoCount>{dr:countVideoSlugs($slugName, $videoFileName)}</availableVideoCount>
     )
};

declare function dr:countVideoSlugs
  ($slugName as xs:string, $videoFileName as xs:string ) as xs:integer{

    count(for $programSerieVideo in doc($videoFileName)//ProgramSerieVideo
          where $programSerieVideo[ProgramSerieSlug = $slugName] and $programSerieVideo/Expired/text() = "false"
          return $programSerieVideo)
};

\end{lstlisting}
\caption{Funktion til at finde antal tilg�ngelige afsnit i en programserie }
\label{joining:checkVideoCount}
\end{figure}

TODO: Insert stuff about joining data

Which XML documents to be joined and why?

Compare the same XML document different dates.
( Is something added,removed,updated ? list results)