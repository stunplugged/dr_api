Det er i projektet lykkes at skabe nogle funktioner, som kan anvendes, som grundl�ggende s�gefunktioner til videoapplikationen. Funktionerne g�re det muligt at lave en dynamisk menu-struktur, hvor kun udvalgte videoer vises, alt efter hvad brugeren har valgt af fx kategori.

I dette projekt kan det konkluderes at XML er et godt sprog til dataudveksling, hvor enkelte objekter, fx i form af detaljer omkring en video eller en programserie skal udveksles. Is�r ses XML sproget som v�rende godt til online verdenen, hvor det ligefrem l�gger op til at outputtet data direkte som HTML. 

XML kommer derimod i problemer n�r der er tale om at holde p� mange informationer med tilh�rende relationer. Her t�nkes fx p� problemstillingen omkring alle programserier i et dokument, og alle informationer omkring de tilh�rende videoer i et andet dokument. Dokumenterne kan uden problemer, indeholde alt informationen, men der findes igen metode til at sikre at relationerne imellem de to dokumenter rent faktisk er eksisterende. En video i XML dokumentet med detaljer om videoerne kan godt referere til en programserie i XML dokumentet omkring alle programserierne, men der er ikke nogen sikkerhed for programserien rent faktisk findes i dette dokument. Her savnes princippet med fremmen�gler fra den relationelle database. Til gengl�de er XML et meget l�sebart sprog for det menneskelige �je, hvor der kan hurtigt skabes et overblik over dataenes hierarkiske struktur, hvilket kan v�re meget vanskeligt n�r der ses p� tabeller i en relationel database.

Selvom XML har en svaghed omkring relationer imellem dokumenter, s� styrkes XML af muligheden for at implementere fx et XML skema til validering af selve dokumentets indhold og struktur. Dette ses som en vigtig egenskab da et XML dokument, som ikke er validt, vil kunne skabe fejl under l�sning, fx hvis en datatype �ndre sig fra at v�re et tal til en tekst. 

Omkring XQuery, s� konkluders det at syntaksen er noget speciel og uvant i forhold til andre sprog. Specielt blev det bem�rket, at det ikke kan lade sig g�re at skrive en IF-s�tning uden, at der skal medtages en ELSE, da denne er p�kr�vet ogs� selvom den ikke skal have noget indhold. 
