Som beskrevet i det foreg�ende kapitel, s� er der ikke i API'et ikke udstillet noget validerings-skema eller nogen indbygget DTD (Document Type Definition) til de XML dokumenter der er til r�dighed. Den manglende mulighed for at validere XML dokumenterne betyder, at videoapplikation m� stole p� at XML dokumenterne altid overholder samme struktur. �ndres strukturen pludseligt vil det betyde at de s�gefunktioner, som udarbejdes i dette projekt, ikke vil kunne fungere l�ngere. Det er dermed sagt, at der kun gives garanti for at alle de funktioner, som udarbejdes i dette projekt kun kan anvendes s�fremt at de XML dokumenter, som hentes fra Danmarks Radio's API, kan valideres imod de XML skemaer, som beskrives i dette kapitel. 

Nogle af fordelene ved at anvende XML skema fremfor DTD er, at XML skema er meget mere kraftfuldt end DTD da XML skema eksempelvis underst�tter datatyper og desuden er selve XML skemaer selv beskrevet med XML Syntax og derfor er relative lette at l�se.

\section{Beskrivelse af XML skema til all.xml}

Skriv om anvendelse af sequence
Skrive om valg af datatyper.
Skriv om fx BroadcastChannel (findes kun i nogle programmer)

\begin{figure}[ht]
%\centering
\begin{lstlisting}[style=XML]
<xs:schema xmlns:xs="http://www.w3.org/2001/XMLSchema">
	<xs:element name="ArrayOfProgramSerieVideo">
		<xs:complexType>
			<xs:sequence>
				<xs:element ref="ProgramSerieVideo" 
				            maxOccurs="unbounded" />
			</xs:sequence>
		</xs:complexType>
	</xs:element>	
	<xs:element name="ProgramSerieVideo">
		<xs:complexType>
			<xs:sequence>
				<xs:element name="Id" 					
				            type="xs:nonNegativeInteger" />
				<xs:element name="Description" 			
				            type="xs:string" />
				<xs:element name="ProgramSerieSlug" 	
				            type="xs:string" />
				<xs:element name="Title" 				
				            type="xs:string" />
				<xs:element name="Duration" 			
				            type="xs:string" />
				<xs:element name="BroadcastTime" 		
				            type="xs:dateTime" 
							nillable="true" />
				<xs:element name="ExpireTime" 			
				            type="xs:dateTime" />
				<xs:element name="PublishTime" 			
				            type="xs:dateTime" />
				<xs:element name="Expired" 				
				            type="xs:boolean" />
				<xs:element name="BroadcastChannel" 
				            type="xs:string" minOccurs="0" maxOccurs="1" />
				<xs:element name="VideoManifestUrl" 	
				            type="xs:anyURI" />
				<xs:element name="VideoResourceUrl" 	
				            type="xs:anyURI" />
				<xs:element name="Premiere" 			
				            type="xs:boolean" />
				<xs:element name="BitrateKbps" 			
				            type="xs:nonNegativeInteger" />
				<xs:element name="Height" 				
				            type="xs:nonNegativeInteger" />
				<xs:element name="Width" 				
				            type="xs:nonNegativeInteger" />
			</xs:sequence>
		</xs:complexType>
	</xs:element>
</xs:schema>
\end{lstlisting}
\caption{XML skema til dokumentet all.xml}
\label{data:all_schema.xsd}
\end{figure}

TODO: Insert stuff about xml schemas

Description of how we create them and the choice of types. 

XML validation testing.