Som beskrevet i det foreg�ende kapitel, s� er der ikke i API'et ikke udstillet noget validerings-skema eller nogen indbygget DTD (Document Type Definition) til de XML dokumenter der er til r�dighed. Den manglende mulighed for at validere XML dokumenterne betyder, at videoapplikation m� stole p� at XML dokumenterne altid overholder samme struktur. �ndres strukturen pludseligt vil det betyde at de s�gefunktioner, som udarbejdes i dette projekt, ikke vil kunne fungere l�ngere. Det er dermed sagt, at der kun gives garanti for at alle de funktioner, som udarbejdes i dette projekt kun kan anvendes s�fremt at de XML dokumenter, som hentes fra Danmarks Radio's API, kan valideres imod de XML skemaer, som beskrives i dette kapitel. 

Nogle af fordelene ved at anvende XML skema fremfor DTD er, at XML skema er meget mere kraftfuldt end DTD da XML skema eksempelvis underst�tter datatyper og desuden er selve XML skemaer selv beskrevet med XML Syntax og derfor er relative lette at l�se.

\section{Beskrivelse af XML skemaerne}
Ved unders�gelse af XML dokumenterne fremgik det at begge XML dokumenter indeholder en sekvens af enten videoer eller programserier. Derfor best�r f�rste del af begge de udarbejde XML skemaer af en sekvens af enten elementet ProgramSerieVideo eller elementet ProgramSerie, som kan forekomme uendeligt mange gange. B�de elementet ProgramSerieVideo og ProgramSerie er oprettet som et complexType element, der best�r af hver sine beskrivende elementer, som fx Slug, Title, Description osv. Disse elementer som beskriver videoen eller programserien er blevet sat til at v�re af en bestemt type. Valget af type til elementerne er fremkommet dels af elements navn og elements indhold, som har givet et hint om hvilke type der er tale om.

Specielt skal det bem�rkes at elementet \ele{BroadcastTime} i �ProgramSerieVideo� er sat til at kunne forekommer i XML dokumentet med en null v�rdi (nillable). Det vil sige at en enkelt video kan have elementet � BroadcastTime � i sig, men uden nogen v�rdi. En anden speciel detalje er at elementet BroadcastChannel i �ProgramSerieVideo�, er sat til at kunne undlades helt i en video. Dette g�lder �vrigt ogs� ShortName, WebCmsImagePath i skemaet til ProgramSeries.

I skemaet til ProgramSeries er det ogs� v�rd at bem�rke element LabelsType, som er et complexType element, som indg�r i et andet complexType element, nemlig elementet ProgramSerieType.  

\begin{figure}[ht]
%\centering
\begin{lstlisting}[style=XML]
<?xml version="1.0"?>
<xs:schema xmlns:xs="http://www.w3.org/2001/XMLSchema">
	<xs:element name="ArrayOfProgramSerieVideo">
		<xs:complexType>
			<xs:sequence>
				<xs:element name="ProgramSerieVideo" 
				            maxOccurs="unbounded"
							type="ProgramSerieVideoType" />
			</xs:sequence>
		</xs:complexType>
	</xs:element>	
	<xs:complexType name="ProgramSerieVideoType">
		<xs:sequence>
			<xs:element name="Id" 					
				        type="xs:nonNegativeInteger" />
			<xs:element name="Description" 			
				        type="xs:string" />
			<xs:element name="ProgramSerieSlug" 	
				        type="xs:string" />
			<xs:element name="Title" 				
				        type="xs:string" />
			<xs:element name="Duration" 			
				        type="xs:string" />
			<xs:element name="BroadcastTime" 		
				        type="xs:dateTime" 
						nillable="true" />
			<xs:element name="ExpireTime" 			
				        type="xs:dateTime" />
			<xs:element name="PublishTime" 			
				        type="xs:dateTime" />
			<xs:element name="Expired" 				
				        type="xs:boolean" />
			<xs:element name="BroadcastChannel" 	
				        type="xs:string" 
						minOccurs="0" 
						maxOccurs="1" />
			<xs:element name="VideoManifestUrl" 	
				        type="xs:anyURI" />
			<xs:element name="VideoResourceUrl" 	
				        type="xs:anyURI" />
			<xs:element name="Premiere" 			
				        type="xs:boolean" />
			<xs:element name="BitrateKbps" 			
				        type="xs:nonNegativeInteger" />
			<xs:element name="Height" 				
				        type="xs:nonNegativeInteger" />
			<xs:element name="Width" 				
				        type="xs:nonNegativeInteger" />
		</xs:sequence>
	</xs:complexType>
</xs:schema>
\end{lstlisting}
\caption{XML skema til dokumentet all.xml}
\label{data:all_schema.xsd}
\end{figure}


\begin{figure}[ht]
%\centering
\begin{lstlisting}[style=XML]
<?xml version="1.0"?>
<xs:schema xmlns:xs="http://www.w3.org/2001/XMLSchema">
	<xs:element name="ArrayOfProgramSerie">
		<xs:complexType>
			<xs:sequence>
				<xs:element name="ProgramSerie" maxOccurs="unbounded" type="ProgramSerieType"/>
			</xs:sequence>
		</xs:complexType>
	</xs:element>
	<xs:complexType name="ProgramSerieType">
		<xs:sequence>
			<xs:element name="Slug" 			type="xs:string" />
			<xs:element name="Title" 			type="xs:string" />
			<xs:element name="Description" 		type="xs:string" />
			<xs:element name="ShortName" 		type="xs:string" minOccurs="0" maxOccurs="1" />
			<xs:element name="NewestVideoId"	type="xs:nonNegativeInteger" />
			<xs:element name="NewestVideoPublishTime" type="xs:dateTime" />
			<xs:element name="VideoCount"		type="xs:nonNegativeInteger" />
			<xs:element name="Labels"			type="LabelsType" />
			<xs:element name="WebCmsImagePath" type="xs:anyURI" minOccurs="0" maxOccurs="1" />
		</xs:sequence>
	</xs:complexType>
	<xs:complexType name="LabelsType">
		<xs:sequence>		
			<xs:element name="string"	type="xs:string" minOccurs="0" maxOccurs="unbounded" />
		</xs:sequence>	
	</xs:complexType>	
</xs:schema>
\end{lstlisting}
\caption{XML skema til dokumentet programseries.xml}
\label{data:programseries.xsd}
\end{figure}

\section{Validering med XML skema}

Valideringen af XML dokumenterne med XML skemaerne er test ved hj�lp af BaseX, hvor validate:xsd er blevet anvendt. For at test XML skemaerne holder over tid, s� er der igennem projektetsforl�b blevet hentet nye versioner af XML dokumenterne, s� ogs� har best�et valideringstesten med de udarbejde XML skemaer. 

\begin{figure}[ht]
%\centering
\begin{lstlisting}[style=FAKE_XQUERY, language=XQUERY]
validate:xsd('all.xml', 'all_schema.xsd')
\end{lstlisting}
\caption{Funktionskald til validering af all.xml med skemaet all\_schema.xsd}
\label{xschema: validering }
\end{figure}
